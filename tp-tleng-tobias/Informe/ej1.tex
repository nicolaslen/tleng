\section{Gramática}

\begin{verbatim}
program
    : statement_list

statement_list
    : COMENTARIO
    | statement_list COMENTARIO
    | statement
    | statement_list statement
    ;

statement
    : expression_statement
    | selection_statement
    | iteration_statement
    | jump_statement
    ;

single_statement
    : comment_list statement
    | statement
    ;

comment_list
    : COMENTARIO comment_list
    | COMENTARIO
    ;

block
    : single_statement
    | bracketed_statement_list
    ;

bracketed_statement_list
    : '{' statement_list '}'
    ;

expression_statement
    : ';'
    | expression ';'
    ;

expression
    : assignment_expression
    | expression ',' assignment_expression
    ;

primary_expression
    : NOMBRE_VARIABLE
    | NUMERO
    | CADENA
    | TRUE
    | FALSE
    | '(' expression ')'
    | '[' vector_expression ']'
    | '{' reg_expression '}'
    | function
    ;

vector_expression
    : conditional_expression
    | conditional_expression ',' vector_expression
    ;

reg_expression
    : NOMBRE_VARIABLE ':' expression
    | NOMBRE_VARIABLE ':' expression, reg_expression
    ;

function
    : MULTIPLICACION_ESCALAR
    | CAPITALIZAR
    | COLINEALES
    | PRINT
    | LENGTH
    ;

postfix_expression
    : primary_expression
    | postfix_expression '[' expression ']'
    | function '(' ')'
    | function '(' argument_expression_list ')'
    | postfix_expression '.' NOMBRE_VARIABLE
    | postfix_expression '++'
    | postfix_expression '--'
    ;

argument_expression_list
    : assignment_expression
    | argument_expression_list ',' assignment_expression
    ;

unary_expression
    : postfix_expression
    | '++' unary_expression
    | '--'  unary_expression
    | unary_operator unary_expression
    ;

unary_operator
    : '+'
    | '-'
    | 'NOT'
    ;

multiplicative_expression
    : unary_expression
    | multiplicative_expression '*' unary_expression
    | multiplicative_expression '/' unary_expression
    | multiplicative_expression '^' unary_expression
    ;

congruence_expression
    : multiplicative_expression
    | congruence_expression '%' multiplicative_expression

additive_expression
    : congruence_expression
    | additive_expression '+' congruence_expression
    | additive_expression '-' congruence_expression
    ;

relational_expression
    : additive_expression
    | relational_expression '<' additive_expression
    | relational_expression '>' additive_expression
    ;

equality_expression
    : relational_expression
    | equality_expression '==' relational_expression
    | equality_expression '!=' relational_expression
    ;

logical_and_expression
    : equality_expression
    | logical_and_expression 'AND' equality_expression
    ;

logical_or_expression
    : logical_and_expression
    | logical_or_expression 'OR' logical_and_expression
    ;

conditional_expression
    : logical_or_expression
    | logical_or_expression '?' expression ':' conditional_expression
    ;

assignment_expression
    : conditional_expression
    | unary_expression '=' assignment_expression
    | unary_expression '*=' assignment_expression
    | unary_expression '/=' assignment_expression
    | unary_expression '+=' assignment_expression
    | unary_expression '-=' assignment_expression
    ;

selection_statement
    : IF '(' expression ')' block
    | IF '(' expression ')' block ELSE block
    ;

iteration_statement
    : WHILE '(' expression ')' block
    | DO block WHILE '(' expression ')' ';'
    | FOR '(' expression_statement expression_statement ')' block
    | FOR '(' expression_statement expression_statement expression ')' block
    ;

jump_statement
    : RETURN ';'
    | RETURN expression ';'
    ;

\end{verbatim}

\section{Notas sobre el trabajo}

El código tiene varios comentarios sobre cómo hicimos el trabajo, las asunciones que tomamos, incluso TODOs con dudas o elementos faltantes, etc.

Por favor lean todos los comentarios, lo esencial a tener en cuenta sobre nuestro código está ahí mismo.

\section{Resumen y conclusión}

Pudimos programar esta gramática gracias a PLY, hacer una TDS para checkear condiciones particulares, y con esto logramos parsear con \textit{prettyprint} y checkeo de tipado básico este lenguaje SLS pseudo C-Python.

En una nota aparte, la programación fue trabajosa y tediosa, no parece nada escalable y no es tan sencillo de leer y seguir. Creemos que es culpa de falta de práctica con esto, un poco de falta de orientación o simplemente porque PLY no es la mejor herramienta y tal vez usar Java con la otra librería era más limpio. De todas formas, los resultados se dieron bien, y el parser se pudo terminar.