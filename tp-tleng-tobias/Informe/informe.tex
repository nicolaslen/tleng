\documentclass[spanish, 10pt,a4paper]{article}
\usepackage[spanish]{babel}
\usepackage[utf8]{inputenc}
\usepackage{textcomp}
\usepackage{hyperref}
\usepackage[pdftex]{graphicx}
\usepackage{epsfig}
\usepackage{amsmath}
\usepackage{hyperref}
\usepackage{amssymb}
\usepackage{color}
\usepackage{graphics}
\usepackage{clrscode3e}
\usepackage{amsthm}
\usepackage{subcaption}
\usepackage{caratula}
\usepackage{fancyhdr,lastpage}
\usepackage[paper=a4paper, left=1.4cm, right=1.4cm, bottom=1.4cm, top=1.4cm]{geometry}
\usepackage[table]{xcolor} % color en las matrices
\usepackage[font=small,labelfont=bf]{caption} % caption de las figuras en letra mas chica que el texto
\usepackage[ruled,vlined,linesnumbered]{algorithm2e}
\usepackage{listings}
\usepackage{float}
\usepackage{amsfonts}
\usepackage{upgreek}


\color{black}

%%%PAGE LAYOUT%%%
\topmargin = -1.2cm
\voffset = 0cm
\hoffset = 0em
\textwidth = 48em
\textheight = 164 ex
\oddsidemargin = 0.5 em
\parindent = 2 em
\parskip = 3 pt
\footskip = 7ex
\headheight = 20pt
\pagestyle{fancy}
\lhead{Tleng - TP SLS} % cambia la parte izquierda del encabezado
\renewcommand{\sectionmark}[1]{\markboth{#1}{}} % cambia la parte derecha del encabezado
\rfoot{\thepage}
\cfoot{}
\numberwithin{equation}{section} %sets equation numbers <chapter>.<section>.<subsection>.<index>

%Lo siguiente controla el ancho de las figuras (principalmente para el texto de los captions)
%\newcommand{\figurewidth}{.9\textwidth}
\newcommand{\figurewidth}{1\textwidth}

\newcommand{\tuple}[1]{\ensuremath{\left \langle #1 \right \rangle }}
\newcommand{\Ode}[1]{\small{$\mathcal{O}(#1)$}}
\newtheorem{teorema}{Teorema}[section]


%El siguiente paquete permite escribir la caratula facilmente
\hypersetup{
  pdftitle={ Tleng },
  colorlinks,
  citecolor=black,
  filecolor=black,
  linkcolor=black,
  urlcolor=black 
}

\materia{Teoría de Lenguajes}

\titulo{Trabajo Práctico SLS: Un simple lenguaje de scripting}

\subtitulo{Informe y análisis de resultados.}

\grupo{Grupo Altokemono}


\integrante{Lebedinsky, Alan}{802/11}{alanlebe@gmail.com}
\integrante{Podavini Rey, Martín Gastón}{483/12}{marto.rey2006@gmail.com}
\integrante{Valdes Castro, Tobías}{800/12}{tobini2@gmail.com}
 
\begin{document}
{ \oddsidemargin = 2em
	\headheight = -20pt
	\maketitle
}
	\tableofcontents
	\newpage
	\section*{Introducción}
\addcontentsline{toc}{section}{Introducción}

Se desea incorporar un lenguaje de scripting, denominado Simple Lenguaje de Scripting (SLS), a
un sistema de software ya existente. Para ello se requiere desarrollar un analizador léxico y sintáctico
para este lenguaje.

Así, se recibirá como entrada un código fuente, el cual se deberá chequear por si cumple la sintaxis y
restricciones de tipado del lenguaje, para, finalmente, formatear el código con la ‘indentación’ adecuada
para SLS. En caso de haberse detectado algún error, se deberá informar claramente cuáles son las
características del mismo.


	\newpage
	\section{Gramática}

\begin{verbatim}
program
    : statement_list

statement_list
    : COMENTARIO
    | statement_list COMENTARIO
    | statement
    | statement_list statement
    ;

statement
    : expression_statement
    | selection_statement
    | iteration_statement
    | jump_statement
    ;

single_statement
    : comment_list statement
    | statement
    ;

comment_list
    : COMENTARIO comment_list
    | COMENTARIO
    ;

block
    : single_statement
    | bracketed_statement_list
    ;

bracketed_statement_list
    : '{' statement_list '}'
    ;

expression_statement
    : ';'
    | expression ';'
    ;

expression
    : assignment_expression
    | expression ',' assignment_expression
    ;

primary_expression
    : NOMBRE_VARIABLE
    | NUMERO
    | CADENA
    | TRUE
    | FALSE
    | '(' expression ')'
    | '[' vector_expression ']'
    | '{' reg_expression '}'
    | function
    ;

vector_expression
    : conditional_expression
    | conditional_expression ',' vector_expression
    ;

reg_expression
    : NOMBRE_VARIABLE ':' expression
    | NOMBRE_VARIABLE ':' expression, reg_expression
    ;

function
    : MULTIPLICACION_ESCALAR
    | CAPITALIZAR
    | COLINEALES
    | PRINT
    | LENGTH
    ;

postfix_expression
    : primary_expression
    | postfix_expression '[' expression ']'
    | function '(' ')'
    | function '(' argument_expression_list ')'
    | postfix_expression '.' NOMBRE_VARIABLE
    | postfix_expression '++'
    | postfix_expression '--'
    ;

argument_expression_list
    : assignment_expression
    | argument_expression_list ',' assignment_expression
    ;

unary_expression
    : postfix_expression
    | '++' unary_expression
    | '--'  unary_expression
    | unary_operator unary_expression
    ;

unary_operator
    : '+'
    | '-'
    | 'NOT'
    ;

multiplicative_expression
    : unary_expression
    | multiplicative_expression '*' unary_expression
    | multiplicative_expression '/' unary_expression
    | multiplicative_expression '^' unary_expression
    ;

congruence_expression
    : multiplicative_expression
    | congruence_expression '%' multiplicative_expression

additive_expression
    : congruence_expression
    | additive_expression '+' congruence_expression
    | additive_expression '-' congruence_expression
    ;

relational_expression
    : additive_expression
    | relational_expression '<' additive_expression
    | relational_expression '>' additive_expression
    ;

equality_expression
    : relational_expression
    | equality_expression '==' relational_expression
    | equality_expression '!=' relational_expression
    ;

logical_and_expression
    : equality_expression
    | logical_and_expression 'AND' equality_expression
    ;

logical_or_expression
    : logical_and_expression
    | logical_or_expression 'OR' logical_and_expression
    ;

conditional_expression
    : logical_or_expression
    | logical_or_expression '?' expression ':' conditional_expression
    ;

assignment_expression
    : conditional_expression
    | unary_expression '=' assignment_expression
    | unary_expression '*=' assignment_expression
    | unary_expression '/=' assignment_expression
    | unary_expression '+=' assignment_expression
    | unary_expression '-=' assignment_expression
    ;

selection_statement
    : IF '(' expression ')' block
    | IF '(' expression ')' block ELSE block
    ;

iteration_statement
    : WHILE '(' expression ')' block
    | DO block WHILE '(' expression ')' ';'
    | FOR '(' expression_statement expression_statement ')' block
    | FOR '(' expression_statement expression_statement expression ')' block
    ;

jump_statement
    : RETURN ';'
    | RETURN expression ';'
    ;

\end{verbatim}

\section{Notas sobre el trabajo}

El código tiene varios comentarios sobre cómo hicimos el trabajo, las asunciones que tomamos, incluso TODOs con dudas o elementos faltantes, etc.

Por favor lean todos los comentarios, lo esencial a tener en cuenta sobre nuestro código está ahí mismo.

\section{Resumen y conclusión}

Pudimos programar esta gramática gracias a PLY, hacer una TDS para checkear condiciones particulares, y con esto logramos parsear con \textit{prettyprint} y checkeo de tipado básico este lenguaje SLS pseudo C-Python.

En una nota aparte, la programación fue trabajosa y tediosa, no parece nada escalable y no es tan sencillo de leer y seguir. Creemos que es culpa de falta de práctica con esto, un poco de falta de orientación o simplemente porque PLY no es la mejor herramienta y tal vez usar Java con la otra librería era más limpio. De todas formas, los resultados se dieron bien, y el parser se pudo terminar.
	\newpage
	\bibliographystyle{plain}
	\clearpage
	\bibliography{bibliography}
	\addcontentsline{toc}{section}{Referencias}
\end{document}

